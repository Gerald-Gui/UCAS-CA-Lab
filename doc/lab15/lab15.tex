\documentclass[UTF-8,twoside,c5size]{ctexart}
\usepackage[dvipsnames]{xcolor}
\usepackage{amsmath}
\usepackage{amssymb}
\usepackage{geometry}
\usepackage{listings}
\usepackage{setspace}
\usepackage{xeCJK}
\usepackage{ulem}
\usepackage{pstricks}
\usepackage{pstricks-add}
\usepackage{bm}
\usepackage{mathtools}
\usepackage{breqn}
\usepackage{mathrsfs}
\usepackage{esint}
\usepackage{textcomp}
\usepackage{upgreek}
\usepackage{pifont}
\usepackage{tikz}
\usepackage{circuitikz}
\usepackage{caption}
\usepackage{tabularx}
\usepackage{array}
\usepackage{pgfplots}
\usepackage{multirow}
\usepackage{pgfplotstable}
\usepackage{mhchem}
\usepackage{graphicx}
\usepackage[cache=false]{minted}
\usepackage{multicol}

\newcolumntype{Y}{>{\centering\arraybackslash}X}
\geometry{a4paper,centering,top=1.27cm,bottom=2.54cm,left=2cm,right=2cm}
\graphicspath{{figures/}}
\pagestyle{plain}
\captionsetup{font=small}

%\CTEXsetup[name={,.}]{section}
\CTEXsetup[format={\raggedright\heiti\noindent\zihao{-3}},numberformat={\bfseries}]{section}
\CTEXsetup[format={\raggedright\heiti\zihao{4}},numberformat={\bfseries}]{subsection}
\CTEXsetup[format={\raggedright\heiti\quad\zihao{-4}},numberformat={\bfseries}]{subsubsection}
\CTEXsetup[format={\raggedright\heiti\qquad},numberformat={\bfseries}]{paragraph}
\CTEXsetup[beforeskip=1.0ex plus 0.2ex minus .2ex, afterskip=1.0ex plus 0.2ex minus .2ex]{paragraph}

\CTEXsetup[format={\raggedright\heiti\qquad},numberformat={\bfseries},name={\bfseries(,)}]{subparagraph}
\CTEXsetup[beforeskip=1.0ex plus 0.2ex minus .2ex, afterskip=1.0ex plus 0.2ex minus .2ex]{subparagraph}
\renewcommand\thefootnote{\ding{\numexpr171+\value{footnote}}}

\setstretch{1.5}

\setCJKfamilyfont{boldsong}[AutoFakeBold = {2.17}]{SimSun}
\newcommand*{\boldsong}{\CJKfamily{boldsong}}
%\DeclareMathOperator\dif{d\!}
\newcommand*{\me}{\mathop{}\!\mathrm{e}}
\newcommand*{\mpar}{\mathop{}\!\partial}
\newcommand*{\dif}{\mathop{}\!\mathrm{d}}
\newcommand*{\tab}{\indent}
\newcommand*{\mcelsius}{\mathop{}\!{^\circ}\mathrm{C}}
\renewcommand*{\Im}{\mathrm{Im}\,}

\setcounter{secnumdepth}{5}

\renewcommand\arraystretch{1.5}
\renewcommand\thesubparagraph{\arabic{subparagraph}}

\lstset{
	backgroundcolor=\color[RGB]{245,245,245},
	keywordstyle=\color{blue}\bfseries,
	basicstyle=\small\ttfamily,
	commentstyle=\itshape\color{olive},
	numberstyle=\ttfamily,
	tabsize=4,
	breaklines=true
}

\setminted{style=manni,fontsize=\small,breaklines=true}

\begin{document}
	\begin{center}
		\heiti\zihao{-2}
		实验\textbf{13 $ \bm\sim $ 15}报告
	\end{center}

	\begin{table*}[!h]
		\raggedleft
		\zihao{-4}
		\begin{tabular}{ccc}
			{\heiti 学号} & {2019K8009929019} & {2019K8009929026} \\
			{\heiti 姓名} & 桂庭辉 & 高梓源 \\
			{\heiti 箱子号} & \multicolumn{2}{c}{44}
		\end{tabular}
	\end{table*}
	
	\section{实验任务}
    
	
	\section{实验设计}	
	\subsection{重要模块设计:\texttt{TLB}}
    \subsubsection{功能描述}
    TLB的功能在于在硬件上记录部分页表项以加快页表查询速度,具体需要记录指定项数的页表项PTE,根据LoongArch 32位精简版TLB需支持的指令,设计同步写、异步读的读写端口以及用于取指、访存两处地址转换的查找功能。
    
    TLB模块的接口与讲义一致,此处不再赘述。
    
    \subsubsection{细节实现}
    
    TLB的读写逻辑与通用寄存器堆regfile的逻辑类似。读操作根据\texttt{r\_index}异步读取TLB记录页表项各域即可,写操作在\texttt{we}拉高时根据\texttt{w\_index}将页表项各域同步更新记录。
    
    查找过程需要将输入的虚页号与TLB记录的虚页号相匹配,生成各表项匹配成功与否的\texttt{match0/1}信号逻辑与讲义一致,具体而言对每个TLB表项需检查存在位\texttt{e}、可见性\texttt{g}或\texttt{asid}匹配、虚页号相等与否,其中虚页号对于页大小为4\,MB时比对高9位即可,而为4\,KB时需匹配所有19位。
    
    \begin{minted}{verilog}
    generate for (i = 0; i < TLBNUM; i = i + 1) begin
        assign match0[i] = tlb_e[i] && (tlb_g[i] || tlb_asid[i] == s0_asid) &&
                            // cmp high bits whatever ps is
                            s0_vppn[18:10] == tlb_vppn[i][18:10]            &&  
                            // if ps == 4 KB, cmp low bits
                           (s0_vppn[ 9: 0] == tlb_vppn[i][ 9: 0] || tlb_ps[i]); 
        assign match1[i] = tlb_e[i] && (tlb_g[i] || tlb_asid[i] == s1_asid) &&
                            s1_vppn[18:10] == tlb_vppn[i][18:10]            &&
                           (s1_vppn[ 9: 0] == tlb_vppn[i][ 9: 0] || tlb_ps[i]);
    end
    endgenerate
    \end{minted}
    
    是否查找成功信号\texttt{found}生成逻辑简单,只需将\texttt{match}按位或(等效于$ \neq 0 $)即可。由\texttt{match}信号获取匹配的TLB表项索引\texttt{index}可以使用多路选择器实现,但在个人设计中未能编写出兼顾代码可读性与参数化(即选择器路数由参数\texttt{TLBNUM}决定)的多路选择器,故而以另一种形式实现该过程,即下文将要讨论的模块\texttt{mylog2}。
    
    获取到查找索引\texttt{s_index}后,由于一个TLB表项记录两个物理页,接下来需要判断奇偶页的选择。标记奇偶页的虚地址位为\texttt{va[PS]},对于4\,KB页,其为虚地址的第12位,对于4\,MB页,其为虚地址的第22页,即查找端口输入的虚页号第9位。根据页大小获取奇偶标志位后即可选出查找结果的物理页。
    
    \begin{minted}{verilog}
    assign s0_ppg_sel = tlb_ps[s0_index] ? s0_vppn[9] : s0_va_bit12;
    assign s0_ps      = tlb_ps[s0_index] ? 6'd22 : 6'd12;

    assign s0_ppn   = s0_ppg_sel ? tlb_ppn1[s0_index] : tlb_ppn0[s0_index];
    assign s0_plv   = s0_ppg_sel ? tlb_plv1[s0_index] : tlb_plv0[s0_index];
    assign s0_mat   = s0_ppg_sel ? tlb_mat1[s0_index] : tlb_mat0[s0_index];
    assign s0_d     = s0_ppg_sel ? tlb_d1  [s0_index] : tlb_d0  [s0_index];
    assign s0_v     = s0_ppg_sel ? tlb_v1  [s0_index] : tlb_v0  [s0_index];
    \end{minted}
    
    接下来考虑INVTLB指令功能的实现,如讲义所言,将其查找逻辑拆分为四个子匹配逻辑:
    \begin{minted}{verilog}
    generate for (i = 0; i < TLBNUM; i = i + 1) begin
       assign inv_match[0][i] = ~tlb_g[i];
       assign inv_match[1][i] =  tlb_g[i];
       assign inv_match[2][i] = s1_asid == tlb_asid[i];
       assign inv_match[3][i] = s1_vppn[18:10] == tlb_vppn[i][18:10]            &&
                               (s1_vppn[ 9: 0] == tlb_vppn[i][ 9: 0] || tlb_ps[i]);
    end        
    endgenerate
    \end{minted}
    
    由子匹配结果容易获得一套各\texttt{invtlb\_op}对应的掩码,INVTLB指令的无效实现只需要将\texttt{op}对应的掩码中为1处的\texttt{tlb\_e}域抹零即可。
    
    \begin{minted}{verilog}
    assign inv_op_mask[0] = 16'b0;
    assign inv_op_mask[1] = 16'b0;
    assign inv_op_mask[2] = inv_match[1];
    assign inv_op_mask[3] = inv_match[0];
    assign inv_op_mask[4] = inv_match[0] & inv_match[2];
    assign inv_op_mask[5] = inv_match[0] & inv_match[2] & inv_match[3];
    assign inv_op_mask[6] = (inv_match[0] | inv_match[2]) & inv_match[3];
    
    always @ (posedge clk) begin
        if (we) begin
            // write logic
        end else begin
            tlb_e <= ~inv_op_mask[invtlb_op] & tlb_e;
        end
    end
    \end{minted}
    
    \subsection{重要模块设计:\texttt{mylog2}}
    \subsubsection{功能描述与设计实现}
    
    设计本意是检测输入中的1,返回其位置。思路可简单概括为折半查找。
    
    折半查找的每一步检索结果与输出结果自高到低每一位一一对应。以32位输入为例,输出应有5位,折半查找第一次判断1落在高16位还是低16位,若高则将输出最高位置1,依此类推,每次折半判断的高(1)低(0)与作为结果的索引一一对应。判断1落在哪一半的实现通过判断高半部分是否有1(是否不等于0),由此在输入有多个1时进行了高位优先,最终获取的结果为最高1的索引,在数值上等同于运算$ \log_2 $。
    
    在具体实现上,对此前提到的高半部分的选取通过语法\texttt{+:}实现,由此需要获取到每次判断的基址。以32位为例,首次基址应为16,判断输入的\texttt{[16+:16]}部分是否有1。类似的对于任意\footnote{实际应用上最好指定位宽\texttt{WIDTH}为2的次幂,非2次幂时\texttt{+:}的越界问题我并未检验与解决。}位宽\texttt{WIDTH}的输入,其索引位宽\texttt{IDX\_WIDTH}为\texttt{WIDTH}对2取对数,首基址只需将$ \mathtt{IDX\_WIDTH}-1 $位置1,其他位置0即可(即对\texttt{WIDTH}数值取半)。
    
    \begin{minted}{verilog}
    localparam IDX_WIDTH = $clog2(WIDTH);
    
    assign base[IDX_WIDTH-1] = {1'b1, {IDX_WIDTH-1{1'b0}}};
    \end{minted}
    
    对于折半查找每层需要做的工作,(1)根据基址判断高半部分是否有1,生成结果\texttt{res}的相应位;(2)计算下一层应使用的基址。前者的实现较为简单。后者归纳下来即高半部分有1则当前层所用基址加上(1)中检测宽度的一半,否则减去。此处的加减法具有一定的特殊性,记当前层对应位索引为\texttt{i},次层对应位为\texttt{i-1}。加数或减数定为在\texttt{i-1}处为1,其他位为0。而被加/减数的\texttt{i}位以后均为0,所以计算结果中\texttt{i-1}处必定为1,根据这个结论可以发现不会出现连续借位的情况,即\texttt{i}位是上一层的\texttt{i-1}位,必定为1可以被借位,若发生加法则该位保持1,若发生减法则被借位置0。由此可以注意到\texttt{base[i]}与\texttt{res[i]}的值是一致的。
    
    \begin{minted}{verilog}
    generate 
        for (i = IDX_WIDTH-1; i > 0; i = i - 1) begin
            assign res[i] = src[base[i]+:({{IDX_WIDTH-1{1'b0}}, 1'b1} << i)] != 0;
            
            for (j = 0; j < IDX_WIDTH; j = j + 1) begin
                if (i == j) begin
                    assign base[i - 1][j] = res[i];
                end else if (i - 1 == j) begin
                    assign base[i - 1][j] = 1'b1;
                end else begin
                    assign base[i - 1][j] = base[i][j]; 
                end                
            end
        end
    endgenerate
    \end{minted}
    
    结果的末位与根据最后一次计算出的基址\texttt{base[0]}索引输入的结果一致。
    
    \begin{minted}{verilog}
    assign res[0] = src[base[0]];
    \end{minted}
    
    \subsubsection{接口定义}
    
    \begin{table}[!h]
        \centering
        \caption{\texttt{mylog2}模块接口定义}
        \begin{tabularx}{0.8\textwidth}{|Y|Y|Y|Y|}
            \hline
            \textbf{名称} & \textbf{方向} & \textbf{位宽} & \textbf{描述} \\
            \hline
            \texttt{src} & \textsc{In} & WIDTH & \multirow{2}*{$ \mathtt{res}=\log_2\mathtt{src} $} \\
            \cline{1-3}
            \texttt{res} & \textsc{Out} & $ \log_2\mathrm{WIDTH} $ & \\
            \hline            
        \end{tabularx}
    \end{table}

	\section{实验过程}
	
	\subsection{实验流水账}
	
	2021.10.28 10:00 $\sim$ 2021.10.28 10:30 :根据讲义内容与gitee仓库tlb\_verify\footnote{https://gitee.com/ucas-ca-edu-lab/tlb_verify}完成TLB初步设计,由于验证环境问题未完全验证设计。
    
    2021.11.18 16:00 $\sim$ 2021.11.18 16:45 :编写设计\texttt{mylog2}模块,根据正式实验环境通过验证。
    
	\subsection{错误记录}	
	
	
	\section{实验总结}
	
	
	
\end{document}